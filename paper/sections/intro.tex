\section{Introduction}
\label{sec:intro}

Most modern  programming languages support dynamic  allocation of heap
data.  Static  analysis of heap data  is much harder  than analysis of
static   and   stack   data.    Garbage   collectors,   for   example,
conservatively  approximate  the liveness  of  heap  objects by  their
reachability  from a  set of  memory  locations called  the {\em  root
  set\/}.  Consequently, many objects  that are reachable but not live
remain  uncollected, causing  a  larger-than-necessary memory  demand.
This is confirmed  by empirical studies on Haskell~\cite{rojemo96lag},
Scheme~\cite{karkare06effectiveness}                                and
Java~\cite{shaham02estimating} programs.

Here we consider a first-order  pure functional language and propose a
liveness  analysis  which  annotates  various program  points  with  a
description  of variables and  fields whose  object references  may be
dereferenced  in the future.   The garbage  collector then  only marks
objects pointed by live references and leaves other, merely reachable,
objects  to  be  reclaimed.    (Although  not  strictly  necessary,  a
collector  would  normally {\em  nullify}  dead  variables and  fields
rather than leaving dangling  references.)  Since there are fewer live
objects   than   reachable   objects,   more  memory   is   reclaimed.
Additionally, since  the collector traverses a smaller  portion of the
heap, the time spent for each collection is also smaller.  The work is
presented in  the context of a  stop-the-world non-incremental garbage
collector (mark-and-sweep,  compacting or  copying) for which  we also
show a  monotonicity result: that  our technique can never  cause more
garbage  collections  to  occur   in  spite  of  changing  the  rather
unpredictable  execution  points   at  which  collections  occur.   We
anticipate that our technique  is applicable to more modern collectors
(generational,  concurrent, parallel),  but leave  such  extensions to
future work.

We first define a {\em fully context-sensitive} (in the sense that its
results  are unaffected  by function  inlining) liveness  analysis and
prove it  correct.  However, fully context-sensitive  methods often do
not scale,  and this analysis would  also require us  to determine, at
run-time,  the internal  liveness of  a  function body  at each  call.
Hence,   similarly   to   the   0-CFA   approach,   we   determine   a
context-independent {\em summary} of  liveness for each function which
safely   approximates   the   context-dependence   of   all   possible
calls~\cite{chatterjee99relevant,cherem07fast,lee05static}.       (Note
that  an intraprocedural context-insensitive  method which  assumes no
information  about function  callers would  be too  imprecise  for our
needs.)   In essence  our approach  sets up  interprocedural data-flow
equations for the liveness summaries  of functions and shows how these
can be solved symbolically  as context-free grammars (CFGs)\@.  We can
then determine  a CFG  for each program  point; these are  then safely
approximated with  finite-state automata  which are encoded  as tables
for each program point.  For  garbage collection purposes only {\em GC
  points}  (program points  representing  calls, to  $\CONS$  or to  a
user-function) need to be stored.
% which may call eventually call $\CONS$, need to be stored.

We  previously proposed  an intraprocedural  method for  heap liveness
analysis   for   a   Java-like   language~\cite{khedker07heap}   which
statically inserted  statements nullifying dead  references to improve
garbage collection; by  contrast nullification here occurs dynamically
(which can work better with  aliasing) when the garbage collector acts
on  liveness  annotations  to  avoid traversing  dead  references.   A
workshop    paper~\cite{karkare07liveness}    outlined    the    basic
0-CFA-style-summary  interprocedural  approach  to  functional-program
liveness  analysis.   The  current  paper adds  the  context-sensitive
analysis and better formalization along with experimental results.
%   to  substantiate our ideas.

\kern -3pt   %% AM: a very small tweak to get back to 2 pages for this
             %% section.
%% AM: rewrite the next paragraph to save 3-4 words and remove this \kern.


%------------------------------------------------------------%
\setcounter{page}{2}
\subsubsection{Motivating Example}
\label{sec:motiv}

Figure~\ref{fig:mot-example}(a) shows  an example program.   The label
$\pi$ of an expression $e$  denotes a program point.  During execution
of  the program, it  represents the  instant of  time just  before the
evaluation  of  $e$.  
%%While we discuss liveness, it  will also be important to consider an
%%instance of time  just after the execution of  an expression, though
%%we shall not be required to mark it with a label.
% memory as  a (possibly  disconnected) directed
% acyclic   graph.\footnote{Since  the   language   under  consideration
%   (Section~\ref{sec:defs}) does not  have any imperative features, the
%   heap cannot  have cycles.}
% Here all data  is assumed to  be boxed, i.e.\
% stored in heap and accessed through references.
We  view the  heap  as a graph.  
Nodes in the heap, also  called  ($\CONS$) {\em  cells\/} contain $\CAR$ and $\CDR$ {\em fields}
containing values.  Edges in the graph are {\em references} and
emanate from {\em variables} or fields.
Variable and field values may also be atomic values ($\NIL$, integers etc.)
While it is convenient to box these in diagrams, our presented
analysis treats them as non-heap values.

Figure~\ref{fig:mot-example}(b)  shows the heap at $\pi$.
The edges shown  by thick arrows are those  which are made live by the
program. In addition,  assuming that
the value of any reachable part of the program result may be explored
or printed, the edges marked by thick dashed arrows are also live.
A cell  is  marked and preserved  during  garbage collection,  only  if it  is
reachable from the  root set through a path of  live edges.  All other
cells can be  reclaimed.  Thus if  a garbage collection takes
% place   at    the   time   represented   by   the    heap   shown   in
at $\pi$ with the    heap   shown   in
Figure~\ref{fig:mot-example}(b), only the cells $\pw$ and $(\CDR~ \pw)$, along with $(\CAR~(\CDR~\pw))$ and all cells reachable from it,
will be marked and preserved.



%%%%%%%%%%%%%%%%%%%%%%%%%%%%%%%%%%%%%%%%%%%%%%%%%%%%%%%%%%%%%%%%%%%%%%%%%%%
%%% MOTIVATING EXAMPLE
\newcommand{\nilfigure}
{\scalebox{0.75}{
\psset{unit=1mm,nodesep=0mm,labelsep=0.5mm}
\begin{pspicture}(0,0)(1,1)
%\psgrid[xunit=1cm,yunit=1cm,gridwidth=.2pt,subgridwidth=.1pt,subgriddiv=5,subgridcolor=gray,gridcolor=blue](0,0)(1,1)
\putnode{start}{origin}{0}{0}{}
\putnode{stop}{origin}{10}{10}{}
\ncline[offsetB=0,nodesepB=0,linewidth=.7]{-}{start}{stop} %here
\end{pspicture}
}}


\begin{figure}[t!]
%%%%{\color{Myblue}
\begin{picture}(100,130)(0,-90)
%  \begin{center}
    \begin{tabular}{cc}
\begin{boxedminipage}{.5\textwidth}
      {\sf
	\renewcommand{\arraystretch}{1}{
	  \begin{uprogram}
	  \UFL\ \hspace*{-.31\TAL} (\DEFINE\ (\append\  \lista\ \listb)
	  \UNL{0}  (\SIF~(\NULLQ \ \lista)
	       \listb
	  \UNL{1}      (\CONS\ (\CAR\  \lista)
%          \UNL{2}          (\CAR\  \lista)
	  \UNL{2}\;\;\;\;\;          (\append\ (\CDR\  \lista)
          \listb))))
          \UNL{0}
	  \UNL{0} \hspace*{-.49\TAL} (\LET\ z  $\leftarrow$(\CONS\ (\CONS\ $4$ (\CONS\ $5$ \NIL))
	  \UNL{3}\ \ \ \  (\CONS\ $6$ \NIL)) \IN
	  \UNL{0} (\LET\ y  $\leftarrow$ (\CONS\ $3$ \NIL) \IN
	  \UNL{1}   (\LET\ w $\leftarrow$ (\append\ y z)\ \IN
	  \UNL{1}               $\,\,\,\,\,\,\,\,\pi$:(\CAR\ (\CDR\ w)))))
	\end{uprogram}
      }} 
\end{boxedminipage}
      &
      \raisebox{-25mm}{\scalebox{.75}{
	%%%%%%%%%%%%%%%%%%%%%Uday's stuff%%%%%%%%%%%%%%%%%%%%%%%%%
      \psset{unit=1mm}
      \psset{linewidth=.3mm}
      \begin{pspicture}(0,-5)(70,60)
	%\psframe(0,0)(73,60)
	%%%%%%%%%%%%%%%%%%%%%%%%%%%%%%%%%%%%%%%%%%%%%%%%%%%%%%%%%%%%%%%%
	\putnode{o}{origin}{13}{50}{\TwoCells{o1}{o2}}
	\putnode{a}{o}{-10}{-15}{\psframebox{3}}
%	\putnode{b}{o}{10}{-15}{\psframebox[linestyle=none,framesep=.5]{\NIL}}
	\ncline[offsetB=-.5,nodesepB=.1]{*->}{o1}{a}
	%\aput[-3.5](.5){\scalebox{1.2}{\psframebox[framesep=.2,linestyle=none,fillstyle=solid,
	%      fillcolor=white]{$\times$}}}
        \putnode{b}{o}{0}{-3}{\psframebox[linestyle=none,framesep=.5]{\scalebox{.63}{\nilfigure}}}
%	\ncline[offsetB=-.5,nodesepB=.1]{*->}{o2}{b}
	%\ncline[offsetB=-.5,nodesepB=.1]{->}{o2}{b}
	\putnode{y}{o}{-14}{8}{\psframebox[linestyle=none,framesep=.5]{y}}
	\nccurve[nodesepB=-.2,angleA=330,angleB=120]{->}{y}{o}
	\aput[-3.5](.5){\scalebox{1.2}{\psframebox[framesep=.2,linestyle=none,fillstyle=solid,
		  fillcolor=white]{$\times$}}}
	%%%%%%%%%%%%%%%%%%%%%%%%%%%%%%%%%%%%%%%%%%%%%%%%%%%%%%
	\putnode{c}{o}{25}{0}{\TwoCells{c1}{c2}}
	\putnode{d}{c}{10}{-10}{\TwoCells{d1}{d2}}
	\putnode{e}{d}{-13}{-12}{\TwoCells{e1}{e2}}
	\putnode{f}{d}{13}{-12}{\TwoCells{f1}{f2}}
	\ncline[nodesepB=-.5]{*->}{c2}{d}
	\ncline[nodesepB=-.5,linewidth=.7]{->}{c2}{d}
	\nccurve[ncurv=1,angleA=270,angleB=330]{*->}{c1}{a}
	\aput[-3.5](.2){\scalebox{1.2}{\psframebox[framesep=.2,linestyle=none,fillstyle=solid,
	      fillcolor=white]{$\times$}}}
	\nccurve[nodesepB=-.5,angleA=240,angleB=70]{*->}{d1}{e}
	\nccurve[nodesepB=-.5,angleA=240,angleB=70,linewidth=.7]{->}{d1}{e}
	\nccurve[nodesepB=-.5,angleA=300,angleB=110]{*->}{d2}{f}
	\aput[-3.5](.5){\scalebox{1.2}{\psframebox[framesep=.2,linestyle=none,fillstyle=solid,
	      fillcolor=white]{$\times$}}}
	\putnode{w}{c}{-8}{8}{\psframebox[linestyle=none,framesep=.2]{w}}
	\putnode{ww}{c}{15}{8}{\psframebox[linestyle=none,framesep=.2]{z}}
	\nccurve[nodesepB=-.2,angleA=330,angleB=120,linewidth=.7]{->}{w}{c}
	%%%%%%%%%%%%%%%%%%%%%%%%%%%%%%%%%%%%%%%%%%%%%%%%%%%%%%%%%%%%%%%%%%
	\putnode{g}{e}{-8}{-12}{\psframebox{4}}
	\putnode{h}{e}{8}{-14}{\TwoCells{h1}{h2}}
	\putnode{i}{f}{-8}{-11}{\psframebox{6}}
%	\putnode{j}{f}{8}{-11}{\psframebox[linestyle=none,framesep=.5]{\NIL}}
	\ncline[offsetB=-.5,nodesepB=.1]{*-}{e1}{e1}
	\ncline[linestyle=dashed,offsetB=-.5,nodesepB=.1,linewidth=.7]{->}{e1}{g}
        %here
        \putnode{j1}{f}{0}{-3}{\psframebox[linestyle=none,framesep=0]{\scalebox{.63}{\nilfigure}}}
        \putnode{j2}{h}{0}{-3}{\psframebox[linestyle=none,framesep=.5]{\scalebox{.63}{\nilfigure}}}

%        \aput[-3.2](.6){\scalebox{1.2}{\psframebox[framesep=.1,linestyle=none,fillstyle=solid,
%	      fillcolor=white]{$\times$}}} %and here
	\ncline[offsetB=-.5,nodesepB=-.3]{*-}{e2}{e2}
	\ncline[linestyle=dashed,offsetB=-.5,nodesepB=.1,linewidth=.7]{->}{e2}{h} 
%	\aput[-3.2](.5){\scalebox{1.2}{\psframebox[framesep=.1,linestyle=none,fillstyle=solid,
%	      fillcolor=white]{$\times$}}} %and here

	\ncline[offsetB=-.5,nodesepB=.1]{*->}{f1}{i}
	\ncline[offsetB=-.5,nodesepB=.1]{*->}{f2}{j}
	\nccurve[nodesepB=-.2,angleA=270,angleB=90]{->}{ww}{d}
	\aput[-3.2](.4){\scalebox{1.2}{\psframebox[framesep=.1,linestyle=none,fillstyle=solid,
	      fillcolor=white]{$\times$}}}
	%%%%%%%%%%%%%%%%%%%%%%%%%%%%%%%%%%%%%%%%%%%%%%%%%%%%%%%%%%%%%%%%%%
	\putnode{k}{h}{-8}{-11}{\psframebox{5}}
%	\putnode{l}{h}{8}{-11}{\psframebox[linestyle=none,framesep=.5]{\NIL}}
	\ncline[offsetB=-.5,nodesepB=.1]{*-}{h1}{h1}
	\ncline[linestyle=dashed,offsetB=-.5,nodesepB=.1,linewidth=.7]{->}{h1}{k}
%	\ncline[offsetB=-.5,nodesepB=.1]{*->}{h2}{l}
	%%%%%%%%%%%%%%%%%%%%%%%%%%%%%%%%%%%%%%%%%%%%%%%%%%%%%%%%%%%%%%%%%%
      \end{pspicture}}} \\
      (a) Example program.&
      \renewcommand{\arraystretch}{.9}{\begin{tabular}[t]{ll}
	(b) Memory graph at $\pi$. Thick edges denote \\
        {\white (b) }live links. Traversal stops at edges marked\\
        {\white (b) }$\times$  during garbage collection.
      \end{tabular}}
    \end{tabular}
\end{picture}

\kern -3ex

\caption{Example Program and its Memory Graph.}\label{fig:mot-example}    
%  \end{center}
%%%%}\color{Myblue} end
\end{figure}


\subsubsection{Organization of the paper}
Section~\ref{sec:defs} gives
the syntax and semantics of
the language used to illustrate our analysis along with basic concepts
and    notations.
% Section~\ref{sec:operational}   describes    the
% operational semantics  of the  language.
Liveness analysis is described in
Section~\ref{sec:liveness} followed by a sketch of a correctness
proof relative to a non-standard semantics.
Section~\ref{sec:computing} shows how to encode liveness as
finite-state automata.
Section~\ref{sec:experiments}   reports   experimental   results and
Section~\ref{sec:lgc-always-better} proves that a liveness based
collector can never do more garbage collections than a reachability
based collector. 
% and Section~\ref{sec:conclusion} concludes the paper.

