Current garbage collectors  leave much heap-allocated data uncollected
because they preserve data {\em reachable} from a root set.
% However, it is only necessary to preserve {\em live} data---a subset of
% reachable data.
However, only  {\em live} data---a subset of  reachable data---need be
preserved.

Using   a    first-order   functional   language    we   formulate   a
context-sensitive liveness  analysis for structured data  and prove it
correct.   We  then use  a  0-CFA-like  conservative approximation  to
annotate  each  allocation  and  function-call program  point  with  a
finite-state  automaton---which   the  garbage-collector  inspects  to
curtail reachability  during marking. As  a result, fewer  objects are
marked  (albeit  with a  more  expensive  marker)  and then  preserved
(e.g.\ by a copy phase).
% and secondly because of the consequent
%reduced number of collections during a computation.

Experiments  confirm the  expected performance  benefits---increase in
garbage  reclaimed  and  a   consequent  decrease  in  the  number  of
collections, a decrease  in the memory size required  to run programs,
and  reduced  overall  garbage  collection  time  for  a  majority  of
programs.
