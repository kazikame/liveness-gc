<<<<<<< HEAD
\section{Liveness}\label{sec:liveness}

In classical liveness analysis a  variable is either `live' (its value
may be  used in future  computation) or `dead' (definitely  not used).
Semantically, a variable is dead at a given program point if arbitrary
changes to its value have no effect on the computation.  Later we will
use $\bot$ to represent a value  which `explodes' when it is used in a
computation; dead variables can  safely have their value replaced with
$\bot$.  For heap-allocated data we need a richer model of liveness in
that both variables  and fields of $\CONS$ cells may  be dead or live.
Using  $\acar$,  $\acdr$  to  represent access  using  $\CAR$,  $\CDR$
respectively, liveness of the structure reachable from a variable is a
set  of  {\em  access  paths}  which  we  represent  as  a  subset  of
$\{\acar,\acdr\}^\ast$, and  use conventional grammar  notation.  Thus
the liveness of $x$ being $\{\acdr\acar, \acdr\acdr\acar\}$ means that
future computation can  only refer to the second  and third members of
$x$   considered  as   a   list.   Semantically,   access  paths   are
prefix-closed, as  accessing a field requires accessing  all the paths
from  the variable  to  the field,  and  hence the  above liveness  is
properly   written   $\{\epsilon,   \acdr,   \acdr\acar,   \acdr\acdr,
\acdr\acdr\acar\}$.  The classical notions  of a scalar variable being
live or dead correspond to $\{\epsilon\}$ and $\{\}$.

The  overall liveness  (also  written {\em  liveness environment}  for
emphasis) at a program point  is conceptually a mapping from variables
to subsets of $\{\acar,\acdr\}^\ast$, but we often abuse notation, for
example writing $\{x.\acar\acdr, x.\acdr, y.\epsilon\}$ instead of the
map           $[x          \mapsto\{\epsilon,\acar,\acar\acdr,\acdr\},
  y\mapsto\{\epsilon\},  z\mapsto\{\}]$.    Analogously  to  classical
liveness,  the liveness  at  program  point $\pi$  in  $\pi:e$ is  the
liveness just before executing $e$.

A complementary notion to liveness is {\em demand}.  The demand for
expression $e$ is again an access path---that subset of of $\{\acar,\acdr\}^\ast$ which the
context of $e$ may explore of $e$'s result.  So, for example given a
demand $\sigma$ and the expression $\pi: (\RETURN\ x)$, the liveness at
$\pi$ is exactly $x.\sigma$.
The classical analogy of this is in {\em strong liveness}, where
an assignment node $n: x:=y+z$ causes $y$ and $z$ to be live on entry
to $n$ if (and only if) $x$ is live at exit of $n$---the liveness
of $x$ at exit from $n$ becomes the demand on $y+z$.
Note that, for an operation like division which
may raise an exception, the assignment $n: x:=y/z$ makes $y$ and $z$ live
regardless of the liveness of $x$.

We use $\sigma$  to range over demands, $\alpha$  to range over access
paths  and $\Lv$  to range  over liveness  environments.  The notation
$\sigma_1\sigma_2$  denotes  the  set $\lbrace  \alpha_1\alpha_2  \mid
\alpha_1 \in  \sigma_1, \alpha_2 \in \sigma_2\rbrace$.  Often we shall
abuse notation to  juxtapose an edge label and a  set of access paths:
$\acar\sigma$   is   a   shorthand  for   $\lbrace\acar\rbrace\sigma$.
Finally,  we use $\Lfonly$  to range  over {\em  demand transformers}.
These transform demands on a function call into demands for its formal
parameters:    given    a    user    function    $f$,    defined    by
$(\DEFINE\  (f\ x_1\  \ldots\ x_n)\  \  e_f)$ and  called with  demand
$\sigma$, then $\Lf{f}{i}{\sigma}$ is the liveness of $x_i$ at $e_f$.

Note that liveness refers to  variables and fields, and not to $\CONS$
cells  (i.e.\  to  edges  in   the  memory  graph,  not  to  locations
themselves).   Hence liveness  of $\{x.\epsilon,x.\acar\}$  means that
future computation may refer to  the value $\ell$ of variable $x$, and
also to the $\CAR$ field of  location $\ell$.  In the absence of other
pointers to heap location $\ell$, we are certain that the $\CDR$ field
of  $\ell$  will  not  be   referenced  and  may  hence  be  corrupted
arbitrarily.   Note therefore,  that  while $\ell$  cannot be  garbage
collected, any location  $\ell'$ stored in the $\CDR$  field of $\ell$
would  be  garbage (again  provided  there  are  no other  aliases  to
$\ell$).



\subsection{Liveness Analysis}
First recall the classical formulation of liveness (as sets of simple variables)
on three-address instructions,
$
\mathit{live}_\mathit{in}(I) = \mathit{live}_\mathit{out}(I)
     \setminus \mathit{def\/}(I)
     \cup   \mathit{ref\/}(I)
$,
and then note that {\em strong liveness} needs, when $I$ is the instruction $z:=x+y$,
that $\mathit{ref}(I)$ be refined to
$\{x,y\}$ if $z \in\mathit{live}_\mathit{out}(I)$ and $\{\}$ otherwise.

Our liveness analysis formulated in Figure~\ref{fig:live-judge}
is analogous.
% has three analogous aspects.
% \begin{itemize}
% \item
Firstly, the function $\mathit{ref}$, when given a statement $s$,
returns the incremental liveness {\em generated} by $s$.
Because we generalize {\em strong} liveness, $\mathit{ref}$ needs a second parameter,
specifying the demand $\sigma$ on the result of $s$,
to determine which access paths of its free variables are made live.
The cases for $(\ID\ x)$ and $(\PRIM\ x\ y)$ exemplify this.
% The cases for
% $(\CAR\ x)$, $(\CAR\ x)$ and $(\CONS\ x\ y)$ show how demand for path $\alpha \in \sigma$
% is transformed into a deeper or shallower path in the demand for (or liveness of) $x$ or $y$.
% The   operator \PRIM\ dereferences its arguments  resulting in the liveness shown.
A demand  of  $\sigma$  on  (\CAR~$x$)  is  transformed  to  the  demand
$\acar\sigma$  on  $x$.   In  addition, \CAR\  always  dereferences  its
argument (even if its result is never used).
This generates the liveness  $\{x.\epsilon\} \cup  x.\acar\sigma$
(note $\sigma$ may be $\{\}$).
In the opposite  sense, the  demand  of $\acar\sigma$  on
(\CONS~$x$~$y$) is  transformed to the  demand $\sigma$ on  $x$.  Note
that \CONS\ does not, by itself, dereference its arguments.
Thirdly, for the case of a user-function call, a third parameter $\Lfonly$ to 
$\mathit{ref}$
expresses how the demand $\sigma$ on the result is transformed into demands on its parameters.
Constants generate no liveness.

% \item
The  function  $\mathcal{L}$ now  gives  the  (total)  liveness of  an
expression $e$.   The cases $\RETURN$ and  $\SIF$ are straightforward,
but note the liveness $x.\epsilon$  generated by the latter.  The case
$(\LET\   z\leftarrow   s\  \IN\   e')$   resembles  a   three-address
instruction:  the liveness  of $e$  is given  by taking  the liveness,
$\Lv$, of $e'$, killing any liveness of $z$ and adding any incremental
liveness from  $s$.  The main subtlety  is how the liveness  of $z$ in
$\Lv$  is converted  to a  demand  $\Lv(z)$ to  be placed  on $s$  via
$\mathit{ref}(s,\Lv(z),\Lfonly)$.

% \item
%%%%%%\marginpar{Say $\mathcal{L}$ is monotonic?}
Finally, the  judgement form $\mathit{Prog}  \len \Lfonly$ is  used to
determine  $\Lfonly$\@.   Analogously   to  classical  liveness  being
computed  as  a  solution  of  dataflow  equations,  we  require,  via
inference   rule  ({\sc  live-define}),   $\Lfonly$  to   satisfy  the
fixed-point property that:  when we assume $\Lfonly$ to  be the family
of demand transformers for the program then the calculated liveness of
each function  body $\mathcal{L}(e_f,\sigma,\Lfonly)$ agrees  with the
assumed $\Lfonly$.   As usual, there  are often multiple  solutions to
$\Lfonly$;  all are  safe (see  Section~\ref{sec:correctness})  but we
prefer  the  least  one  as  giving  the  least  liveness  subject  to
safety---and hence greatest amount of garbage collected.
% \end{itemize}

We make three observations: firstly the rule ({\sc live-define}) has a
least  solution  as  $\mathcal{L}(\cdot)$  is monotonic  in  $\sigma$;
secondly  that  ({\sc  live-define})   resembles  the  rule  for  type
inference of mutually recursive  function definitions, and thirdly the
asymmetry of  demand and liveness (compared to  post- and pre-liveness
classically) is due to the functional formulation here.

Section~\ref{sec:computing} shows how the demand transformers $\Lfonly$ for a program
(representing a fully context-sensitive analysis)
can be safely approximated, for each function, by a {\em procedure summary}
(unifying the contexts in the style of 0-CFA)\@.
The summary consists of a pair of a single demand and, for this demand,
the corresponding tuple of demands the function makes on its arguments.

%==============================================================
\begin{figure}[t]  
% \footnotesize
\begin{eqnarray*}
\mathit{ref\/}(\kappa,\sigma,\Lfonly)
          &=& \{\,\} \mbox{, for $\kappa$ a constant, including $\NIL$}\\
\mathit{ref\/}((\CONS~x~y),\sigma,\Lfonly)
          &=& \{x.\alpha \mid \acar\alpha \in \sigma\} \cup \{y.\alpha \mid \acdr\alpha \in \sigma\} \\
\mathit{ref\/}((\CAR~x),\sigma,\Lfonly)
          &=&    \begin{array}{l l}
                    \{x.\epsilon\} \cup \{x.\acar\alpha \mid \alpha \in
\sigma\}, & \mbox{if}~\sigma \ne \emptyset\\
                    \emptyset  & \mbox{otherwise}
                 \end{array} \\
\mathit{ref\/}((\CDR~x),\sigma,\Lfonly)
          &=&    \begin{array}{l l}
                    \{x.\epsilon\} \cup \{x.\acdr\alpha \mid \alpha \in
\sigma\}, & \mbox{if}~\sigma \ne \emptyset\\
                    \emptyset  & \mbox{otherwise}
                 \end{array} \\
\mathit{ref\/}((\PRIM~x~y),\sigma,\Lfonly)
          &=&    \begin{array}{l l}
                    \{x.\epsilon, y.\epsilon\},  & \mbox{if}~\sigma \ne \emptyset\\
                    \emptyset  & \mbox{otherwise}
                 \end{array} \\
\mathit{ref\/}((\NULLQ~x),\sigma,\Lfonly)
          &=&    \begin{array}{l l}
                    \{x.\epsilon\},  & \mbox{if}~\sigma \ne \emptyset\\
                    \emptyset  & \mbox{otherwise}
                 \end{array} \\
\mathit{ref\/}((f~\myvec{x}),\sigma,\Lfonly)
%          &=& \bigcup_{i=1}^n y_i.\Lf{f}{i}{\sigma}
          &=&  \begin{array}{@{}l}  % to discourage \displaystyle
               \bigcup_{i=1}^n x_i.\Lf{f}{i}{\sigma}
               \end{array}
%          &=& \bigcup \{y_i.\Lf{f}{i}{\sigma} \mid i=1,\ldots, n\}
\\[1ex]
\mathcal{L}((\RETURN~x),\sigma,\Lfonly) &=& x.\sigma \\
\mathcal{L}((\SIF~x~e_1~e_2),\sigma,\Lfonly) &=&
        \begin{array}{l l}
                    \mathcal{L}(e_1,\sigma,\Lfonly) \cup
        \mathcal{L}(e_2,\sigma,\Lfonly) \cup
        \{x.\epsilon\},  & \mbox{if}~\sigma \ne \emptyset\\
                    \mathcal{L}(e_1,\sigma,\Lfonly) \cup
        \mathcal{L}(e_2,\sigma,\Lfonly)  & \mbox{otherwise}
                 \end{array} \\
\mathcal{L}(\LET~\myvec{x} \leftarrow \myvec{s}}~\IN~e),\sigma,\Lfonly) &=&
        \Lv
           \setminus \{\myvec{x}.*\}
           \cup LE(L)
\mbox{ where } \Lv = \lfloor\mathcal{L}(e,\sigma,\Lfonly)\rfloor_{\myvec{x}}
\end{eqnarray*}

\begin{minipage}{0.85\textwidth}
\infrule[live-define]
        {\mathcal{L}(e_f,\sigma,\Lfonly) = 
           \bigcup_{i=1}^n z_i.\Lf{f}{i}{\sigma}
              \mbox{ for each $f$ and $\sigma$}
        }
        { d_1 \ldots d_k \len \Lfonly
\\ \makebox[0mm]{where
     $(\DEFINE\ (f\ z_1\ \ldots\ z_n)\ \ e_f)$ is a member of $d_1 \ldots d_k$}}
\end{minipage}
\begin{minipage}{0.85\textwidth}
\infrule[live-define]
        {\Le{L} = \biguplus_{i=1}^n \Le{ref(s_i, L(x_i), \Lfonly)}
              \mbox{ for each $L$}
        }
        { \myvec{x} \leftarrow \myvec{s} \len \Leonly
\\ \makebox[0mm]{where
     $(\DEFINE\ (f\ z_1\ \ldots\ z_n)\ \ e_f)$ is a member of $d_1 \ldots d_k$}}
\end{minipage}
\normalsize
  \caption{Liveness equations and judgement rule}\label{fig:live-judge}
\end{figure}
=======
\section{Liveness}\label{sec:liveness}
%\comment{
To describe the issues that arise while formulating liveness of a lazy
language, consider the example shown in Figure ???

%%%%%%%%%%%%%%%%%%%%%%%%%%%%%%%%%
\begin{verbatim}
pi_0: let pi_1: x = (cons 1 x)
      in let  pi_2: y = (cdr x)
         in  let  pi_3: z = (car y)
             in pi_4: if (z > 0) then 
                       pi_5: let pi_6: w = (take 5 x)
                             in let pi_7: u = (cdr  w)
                                in l1:return u
                      else l2:return y

\end{verbatim}
%%%%%%%%%%%%%%%%%%%%%%%%%%%%%%%%%% 

The program is annotated with symbols standing for program points. For
reasons that will  become clear later, we use  different symbols for
the program points corresponding to return points (l instead of $\pi$)
Similar  to classical  liveness  analysis for  imperative programs,  a
demand  of  $\sigma$ on  the  expression  propagates  to a  demand  of
$\sigma$  on the return  values $u$  and $y$.  This gives  liveness of
$\lbrace u.\sigma \rbrace$ and $\{y.\sigma\}$ for the program points l1 and l2. We now
show how the liveness at l1 is propagated backwards; l2 can be treated
similarly.

Since u is (cdr w), a liveness of $\sigma$ for u translates into a
demand of $\sigma$ on $(cdr~w)$. This is turn translates into a liveness
of $\{\epsilon,~1.\sigma\}$ of $w$. In the case of eager evaluation, we
would have the livenesses of $\pi_7$ and l1 as $\{\epsilon,~1.\sigma\}$ and
${u.\sigma}$. The liveness of $u$ would have been killed at $\pi_7$. The
liveness of lazy evaluation differs in the following two respects:
\begin{enumerate}
\item The definition  $u = (cdr~w)$, does not  result in an evaluation
  of $(cdr~w)$; rather a  closure is created.  The evaluation actually
  takes place  while evaluating $(return~u)$. Thus it  is necessary to
  record the liveness of $\{\epsilon,  1.\sigma\}$ for $w$ also at the
  program point l1.
  
  Regard the  expression as a  tree and consider  paths from
  the  root  to   program  points  corresponding  to  return
  expressions.   The  above discussion  shows  that there  a
  single liveness  environment for each  such path. Further,
  if a program  point happens to fall on  more than one such
  path, then  the liveness environment of  the program point
  is the variable-wise union of the liveness environments of
  the individual paths.
\item The second difference is that since our lets are lazy,
  the let  definition can be  recursive as at  program point
  $\pi_1$. In such a case, the liveness has to be formulated
  as a fix-point computation.
\end{enumerate}

We therefore  formulate the per-program-point  liveness that
we require for garbage collection in terms of two ....

A  map $\cal{M}$  that gives  the common  liveness  for each
return path.

A  map that  gives  the  return paths  that  pass through  a
program point.

 

%}
In classical liveness analysis a  variable is either `live' (its value
may be  used in future  computation) or `dead' (definitely  not used).
Semantically, a variable is dead at a given program point if arbitrary
changes to its value have no effect on the computation.  Later we will
use $\bot$ to represent a value  which `explodes' when it is used in a
computation; dead variables can  safely have their value replaced with
$\bot$.  For heap-allocated data we need a richer model of liveness in
that both variables  and fields of $\CONS$ cells may  be dead or live.
Using  $\acar$,  $\acdr$  to  represent access  using  $\CAR$,  $\CDR$
respectively, liveness of the structure reachable from a variable is a
set  of  {\em  access  paths}  which  we  represent  as  a  subset  of
$\{\acar,\acdr\}^\ast$, and  use conventional grammar  notation.  Thus
the liveness of $x$ being $\{\acdr\acar, \acdr\acdr\acar\}$ means that
future computation can  only refer to the second  and third members of
$x$   considered  as   a   list.   Semantically,   access  paths   are
prefix-closed, as  accessing a field requires accessing  all the paths
from  the variable  to  the field,  and  hence the  above liveness  is
properly   written   $\{\epsilon,   \acdr,   \acdr\acar,   \acdr\acdr,
\acdr\acdr\acar\}$.  The classical notions  of a scalar variable being
live or dead correspond to $\{\epsilon\}$ and $\{\}$.

The  overall liveness  (also  written {\em  liveness environment}  for
emphasis) at a program point  is conceptually a mapping from variables
to subsets of $\{\acar,\acdr\}^\ast$, but we often abuse notation, for
example writing $\{x.\acar\acdr, x.\acdr, y.\epsilon\}$ instead of the
map           $[x          \mapsto\{\epsilon,\acar,\acar\acdr,\acdr\},
  y\mapsto\{\epsilon\},  z\mapsto\{\}]$.    Analogously  to  classical
liveness,  the liveness  at  program  point $\pi$  in  $\pi:e$ is  the
liveness just before executing $e$.

A complementary notion to liveness is {\em demand}.  The demand for
expression $e$ is again an access path---that subset of of $\{\acar,\acdr\}^\ast$ which the
context of $e$ may explore of $e$'s result.  So, for example given a
demand $\sigma$ and the expression $\pi: (\RETURN\ x)$, the liveness at
$\pi$ is exactly $x.\sigma$.
The classical analogy of this is in {\em strong liveness}, where
an assignment node $n: x:=y+z$ causes $y$ and $z$ to be live on entry
to $n$ if (and only if) $x$ is live at exit of $n$---the liveness
of $x$ at exit from $n$ becomes the demand on $y+z$.
Note that, for an operation like division which
may raise an exception, the assignment $n: x:=y/z$ makes $y$ and $z$ live
regardless of the liveness of $x$.

We use $\sigma$  to range over demands, $\alpha$  to range over access
paths  and $\Lv$  to range  over liveness  environments.  The notation
$\sigma_1\sigma_2$  denotes  the  set $\lbrace  \alpha_1\alpha_2  \mid
\alpha_1 \in  \sigma_1, \alpha_2 \in \sigma_2\rbrace$.  Often we shall
abuse notation to  juxtapose an edge label and a  set of access paths:
$\acar\sigma$   is   a   shorthand  for   $\lbrace\acar\rbrace\sigma$.
Finally,  we use $\Lfonly$  to range  over {\em  demand transformers}.
These transform demands on a function call into demands for its formal
parameters:    given    a    user    function    $f$,    defined    by
$(\DEFINE\  (f\ x_1\  \ldots\ x_n)\  \  e_f)$ and  called with  demand
$\sigma$, then $\Lf{f}{i}{\sigma}$ is the liveness of $x_i$ at $e_f$.

Note that liveness refers to  variables and fields, and not to $\CONS$
cells  (i.e.\  to  edges  in   the  memory  graph,  not  to  locations
themselves).   Hence liveness  of $\{x.\epsilon,x.\acar\}$  means that
future computation may refer to  the value $\ell$ of variable $x$, and
also to the $\CAR$ field of  location $\ell$.  In the absence of other
pointers to heap location $\ell$, we are certain that the $\CDR$ field
of  $\ell$  will  not  be   referenced  and  may  hence  be  corrupted
arbitrarily.   Note therefore,  that  while $\ell$  cannot be  garbage
collected, any location  $\ell'$ stored in the $\CDR$  field of $\ell$
would  be  garbage (again  provided  there  are  no other  aliases  to
$\ell$).



\subsection{Liveness Analysis}
First recall the classical formulation of liveness (as sets of simple variables)
on three-address instructions,
$
\mathit{live}_\mathit{in}(I) = \mathit{live}_\mathit{out}(I)
     \setminus \mathit{def\/}(I)
     \cup   \mathit{ref\/}(I)
$,
and then note that {\em strong liveness} needs, when $I$ is the instruction $z:=x+y$,
that $\mathit{ref}(I)$ be refined to
$\{x,y\}$ if $z \in\mathit{live}_\mathit{out}(I)$ and $\{\}$ otherwise.

Our liveness analysis formulated in Figure~\ref{fig:live-judge}
is analogous.
% has three analogous aspects.
% \begin{itemize}
% \item
Firstly, the function $\mathit{ref}$, when given a statement $s$,
returns the incremental liveness {\em generated} by $s$.
Because we generalize {\em strong} liveness, $\mathit{ref}$ needs a second parameter,
specifying the demand $\sigma$ on the result of $s$,
to determine which access paths of its free variables are made live.
The cases for $(\ID\ x)$ and $(\PRIM\ x\ y)$ exemplify this.
% The cases for
% $(\CAR\ x)$, $(\CAR\ x)$ and $(\CONS\ x\ y)$ show how demand for path $\alpha \in \sigma$
% is transformed into a deeper or shallower path in the demand for (or liveness of) $x$ or $y$.
% The   operator \PRIM\ dereferences its arguments  resulting in the liveness shown.
A demand  of  $\sigma$  on  (\CAR~$x$)  is  transformed  to  the  demand
$\acar\sigma$  on  $x$.   In  addition, \CAR\  always  dereferences  its
argument (even if its result is never used).
This generates the liveness  $\{x.\epsilon\} \cup  x.\acar\sigma$
(note $\sigma$ may be $\{\}$).
In the opposite  sense, the  demand  of $\acar\sigma$  on
(\CONS~$x$~$y$) is  transformed to the  demand $\sigma$ on  $x$.  Note
that \CONS\ does not, by itself, dereference its arguments.
Thirdly, for the case of a user-function call, a third parameter $\Lfonly$ to 
$\mathit{ref}$
expresses how the demand $\sigma$ on the result is transformed into demands on its parameters.
Constants generate no liveness.

% \item
The  function  $\mathcal{L}$ now  gives  the  (total)  liveness of  an
expression $e$.   The cases $\RETURN$ and  $\SIF$ are straightforward,
but note the liveness $x.\epsilon$  generated by the latter.  The case
$(\LET\   z\leftarrow   s\  \IN\   e')$   resembles  a   three-address
instruction:  the liveness  of $e$  is given  by taking  the liveness,
$\Lv$, of $e'$, killing any liveness of $z$ and adding any incremental
liveness from  $s$.  The main subtlety  is how the liveness  of $z$ in
$\Lv$  is converted  to a  demand  $\Lv(z)$ to  be placed  on $s$  via
$\mathit{ref}(s,\Lv(z),\Lfonly)$.

% \item
%%%%%%\marginpar{Say $\mathcal{L}$ is monotonic?}
Finally, the  judgement form $\mathit{Prog}  \len \Lfonly$ is  used to
determine  $\Lfonly$\@.   Analogously   to  classical  liveness  being
computed  as  a  solution  of  dataflow  equations,  we  require,  via
inference   rule  ({\sc  live-define}),   $\Lfonly$  to   satisfy  the
fixed-point property that:  when we assume $\Lfonly$ to  be the family
of demand transformers for the program then the calculated liveness of
each function  body $\mathcal{L}(e_f,\sigma,\Lfonly)$ agrees  with the
assumed $\Lfonly$.   As usual, there  are often multiple  solutions to
$\Lfonly$;  all are  safe (see  Section~\ref{sec:correctness})  but we
prefer  the  least  one  as  giving  the  least  liveness  subject  to
safety---and hence greatest amount of garbage collected.
% \end{itemize}

We make three observations: firstly the rule ({\sc live-define}) has a
least  solution  as  $\mathcal{L}(\cdot)$  is monotonic  in  $\sigma$;
secondly  that  ({\sc  live-define})   resembles  the  rule  for  type
inference of mutually recursive  function definitions, and thirdly the
asymmetry of  demand and liveness (compared to  post- and pre-liveness
classically) is due to the functional formulation here.

Section~\ref{sec:computing} shows how the demand transformers $\Lfonly$ for a program
(representing a fully context-sensitive analysis)
can be safely approximated, for each function, by a {\em procedure summary}
(unifying the contexts in the style of 0-CFA)\@.
The summary consists of a pair of a single demand and, for this demand,
the corresponding tuple of demands the function makes on its arguments.

%==============================================================
\begin{figure}[t]  
% \footnotesize
\begin{eqnarray*}
\mathit{ref\/}(\kappa,\sigma,\Lfonly)
          &=& \{\,\} \mbox{, for $\kappa$ a constant, including $\NIL$}\\
\mathit{ref\/}((\CONS~x~y),\sigma,\Lfonly)
          &=& \{x.\alpha \mid \acar\alpha \in \sigma\} \cup \{y.\alpha \mid \acdr\alpha \in \sigma\} \\
\mathit{ref\/}((\CAR~x),\sigma,\Lfonly)
          &=&    \begin{array}{l l}
                    \{x.\epsilon\} \cup \{x.\acar\alpha \mid \alpha \in
\sigma\}, & \mbox{if}~\sigma \ne \emptyset\\
                    \emptyset  & \mbox{otherwise}
                 \end{array} \\
\mathit{ref\/}((\CDR~x),\sigma,\Lfonly)
          &=&    \begin{array}{l l}
                    \{x.\epsilon\} \cup \{x.\acdr\alpha \mid \alpha \in
\sigma\}, & \mbox{if}~\sigma \ne \emptyset\\
                    \emptyset  & \mbox{otherwise}
                 \end{array} \\
\mathit{ref\/}((\PRIM~x~y),\sigma,\Lfonly)
          &=&    \begin{array}{l l}
                    \{x.\epsilon, y.\epsilon\},  & \mbox{if}~\sigma \ne \emptyset\\
                    \emptyset  & \mbox{otherwise}
                 \end{array} \\
\mathit{ref\/}((\NULLQ~x),\sigma,\Lfonly)
          &=&    \begin{array}{l l}
                    \{x.\epsilon\},  & \mbox{if}~\sigma \ne \emptyset\\
                    \emptyset  & \mbox{otherwise}
                 \end{array} \\
\mathit{ref\/}((f~\myvec{x}),\sigma,\Lfonly)
%          &=& \bigcup_{i=1}^n y_i.\Lf{f}{i}{\sigma}
          &=&  \begin{array}{@{}l}  % to discourage \displaystyle
               \bigcup_{i=1}^n x_i.\Lf{f}{i}{\sigma}
               \end{array}
%          &=& \bigcup \{y_i.\Lf{f}{i}{\sigma} \mid i=1,\ldots, n\}
\\[1ex]
\mathcal{L}((\RETURN~x),\sigma,\Lfonly) &=& \{l \mapsto x.\sigma\}, \makebox[0mm]{\hspace*{3cm} where $l$ is a new label} \\
\mathcal{L}((\SIF~x~e_1~e_2),\sigma,\Lfonly) &=&
        \begin{array}{l l}
                    \mathcal{L}(e_1,\sigma,\Lfonly) \uplus
        \mathcal{L}(e_2,\sigma,\Lfonly) \uplus
        \{l \mapsto  \{x.\epsilon\}\},  & \mbox{if}~\sigma \ne \emptyset\\
        \emptyset  & \mbox{otherwise}
                 \end{array} \\
 \makebox[0mm]{\hspace*{3.8cm} where $l$ is a new label}\\
\mathcal{L}(\LET~x \leftarrow~s~\IN~e),\sigma,\Lfonly) &=&
        \{ l \mapsto \Ldonly_{x \leftarrow s}(\Lv(x)) \mid l \mapsto \Lv
        \in \mathcal{M}\}
\mbox{ where } \mathcal{M} = \mathcal{L}(e,\sigma,\Lfonly)\\
\Ldonly_{x \leftarrow s} &=& \Lfonly_f, \mbox{where}~f~\mbox{is a new function}\\
 \makebox[0mm]{\hspace*{7cm}   
 $(\DEFINE~(f\,x_1\,x_2\, \ldots \, x_n)~(\SIF\,*~x~s[(f\,x_1\,
           x_2\, \ldots\, x_n)/x]))$,} \\
 \makebox[0mm]{\hspace*{5.5cm} and 
     $(\ x_1\, \ldots\, x_n)$ are the variables in $s$}
\end{eqnarray*}

\begin{minipage}{0.85\textwidth}
\infrule[live-define]
        {\mathcal{L}(e_f,\sigma,\Lfonly) = 
           \bigcup_{i=1}^n z_i.\Lf{f}{i}{\sigma}
              \mbox{ for each $f$ and $\sigma$}
        }
        { d_1 \ldots d_k \len \Lfonly
\\ \makebox[0mm]{where
     $(\DEFINE\ (f\ z_1\ \ldots\ z_n)\ \ e_f)$ is a member of $d_1 \ldots d_k$}}
\end{minipage}

\bigskip
\begin{center}
\begin{minipage}{0.85\textwidth}
\end{minipage}
\end{center}
%% \begin{minipage}{0.85\textwidth}
%%  \infrule[live-let-define]
%%          {f~\stackrel{\text{\tiny def}}{=}~(\DEFINE~(f~x_1
%%            ~x_2 \ldots x_n)~(\SIF~*~x~s[(f~x_1~
%%            x_2\ldots x_n)/x]),~\mbox{and}~\Ldonly_{x \leftarrow s}~\stackrel{\text{\tiny def}}{=}~\Lfonly_f}       
%%          { x \leftarrow s \len \Ldonly_{x \leftarrow s}
%% \\ \makebox[0mm]{where
%%      $(\ x_1\ \ldots\ x_n)$ are the variables in $s$}}
%% \end{minipage}
\normalsize
  \caption{Liveness equations and judgement rule}\label{fig:live-judge}
\end{figure}
>>>>>>> 4bb03a5c3d7b9f8e48b056d77b2ef7801e7effa3
