We circumvent the problem of  undecidability by over approximating the
CFG by non-deterministic finite state automata (NFA) using
Mohri   and   Nederhof method~\cite{mohri00regular}.   This   method
transforms a CFG $G$ into a strongly  regular grammar $R$ such that $\lang{G} \subseteq \lang{R}$. 
%%
If a CFG consists of a set of mutually recursive non-terminals such that the rules involved are not all left regular or not all right regular, then the method breaks the rules into right regular rules by introducing fresh non-terminals. For our example, the rule  $\var{\Df{\length}{1}}$ has two non regular productions,           $\acdr\var{\Df{\length}{1}}\clazy$          and
$\clazy\var{\Df{\length}{1}}{\clazy}$.  The steps for transforming these productions into
right regular productions are:

\begin{enumerate}
\item
Add a  new non-terminal  $\varnew{\Df{\length}{1}}$ to the  grammar with
the rule $\varnew{\Df{\length}{1}} \rightarrow \epsilon$.
\item
 Replace     $\var{\Df{\length}{1}}     \rightarrow     \clazy$     by
 $\var{\Df{\length}{1}} \rightarrow \clazy\varnew{\Df{\length}{1}}$
\item
 Replace
 $\var{\Df{\length}{1}} \rightarrow \clazy\var{\Df{\length}{1}}\clazy$
 by  $\var{\Df{\length}{1}}  \rightarrow  \clazy\var{\Df{\length}{1}}$
 and \linebreak $\varnew{\Df{\length}{1}} \rightarrow \clazy\varnew{\Df{\length}{1}}$
\item
  Replace
 $\var{\Df{\length}{1}} \rightarrow \acdr\var{\Df{\length}{1}}\clazy$
 by $\var{\Df{\length}{1}} \rightarrow \acdr\var{\Df{\length}{1}}$ and
 \linebreak $\varnew{\Df{\length}{1}} \rightarrow \clazy\varnew{\Df{\length}{1}}$
\end{enumerate}

Mohri and Nederhof~\cite{mohri00regular} provide the generalization of
these  steps.  The  rules for  
$\var{\Df{\length}{1}}$ after the transformation are:
\begin{eqnarray*}              
  \var{\Df{\length}{1}}   &\rightarrow&   \clazy\varnew{\Df{\length}{1}}
  \mid                 \acdr\var{\Df{\length}{1}}                 \mid
  \clazy\var{\Df{\length}{1}}\\  \varnew{\Df{\length}{1}}  &\rightarrow&
  \clazy\varnew{\Df{\length}{1}} \mid \epsilon
\end{eqnarray*}



The strongly regular grammar is converted  into a set of NFAs, one for
each non-terminal.  The  $\hookrightarrow$ simplification is now
done on the NFAs by  repeatedly introducing $\epsilon$ edges to bypass
pairs  of consecutive  edges  labeled \bcar\acar\  or \bcdr\acdr\  and
constructing  the $\epsilon$-closure  till a  fixed point  is reached,
after which  the edges labeled  \bcar \  and \bcdr\ are  deleted.  The
details of the  algorithm, its correctness and  termination proofs are
given in~\cite{karkare07liveness,asati14lgc}.

The resulting automaton has edges labeled with \acar, \acdr\ and
\clazy\ only.  In this  automaton, for every  edge labeled  \clazy, we
check if the source node of the edge  has a path to a final state.  If
it does, we mark the source node  as final. Finally, we remove all the
edges labeled \clazy\  and convert the automaton  into a deterministic
automaton.   These steps  effectively implement  the $\hookrightarrow$
simplification rules for \bcar, \bcdr, and
\clazy\ to  obtain forward access paths.  While checking for  liveness during
garbage collection,  a forward  access path  is valid  only if  it can
reach a final state.

\newcommand{\nfaD}{%
  \scalebox{.7}{
    \psset{unit=1mm,nodesep=0mm,labelsep=0.5mm}
    \begin{pspicture}(3,-2)(28,10)
      \putnode{t0}{origin}{2}{-1}{} 
      \putnode{t1}{t0}{8}{0}{\pscirclebox{\mbox{\ \ \ \ }}}
      \hspace{5mm} 
      \putnode{t2}{t1}{15}{0}{\pscirclebox[doubleline=true]{\mbox{\ \ \ \ }}} 
      \psset{arrows=->} \ncline{t0}{t1} 
      \ncline{t1}{t2} 
      \putnode{l0}{t1}{7}{2}{\clazy} 
      \nccurve[angleA=45, angleB=135, ncurv=3, nodesep=-1]{t1}{t1} 
      \putnode{l1}{t1}{0}{7}{\acdr, \clazy} 
      \nccurve[angleA=45,
        angleB=135, ncurv=2, nodesep=-1]{t2}{t2} 
      \putnode{l2}{t2}{0}{7}{\clazy}
    \end{pspicture}
}}
\newcommand{\dfaD}{%
  \scalebox{.7}{
    \psset{unit=1mm,nodesep=0mm,labelsep=0.5mm}
    \begin{pspicture}(2,-2)(13,8)
      \putnode{t0}{origin}{2}{0}{}
      \putnode{t1}{t0}{8}{0}{\pscirclebox[doubleline=true]{\mbox{\ \ \ \ }}}     
      \psset{arrows=->}    
      \ncline{t0}{t1}     
      \nccurve[angleA=45, angleB=135, ncurv=2, nodesep=-1]{t1}{t1} 
      \putnode{l1}{t1}{0}{7}{\acdr}
    \end{pspicture}
}}
The  NFA for \var{\Df{\length}{1}} is 
\nfaD\ and the final DFA is \dfaD.
This  expectedly says  that
for  a  demand  $\sigma_{\length}$,   the  liveness  of  the  argument
of \length\ is $\acdr^{*}$ (the spine of the list is traversed).


\newcommand{\dfaE}{%
  \scalebox{.7}{
    \psset{unit=1mm,nodesep=0mm,labelsep=0.5mm}
    \begin{pspicture}(0,-2)(35,8)
     \putnode{t0}{origin}{0}{0}{} 
     \putnode{t1}{t0}{8}{0}{\pscirclebox{\mbox{\ \ \ \
      }}} 
     \hspace{5mm} 
   \putnode{t2}{t1}{12}{0}{\pscirclebox{\mbox{\  \ \ \ }}}
   \psset{arrows=->} \ncline{t0}{t1} \ncline{t1}{t2} 
   \putnode{l0}{t1}{5}{2}{\clazy}
   \putnode{tD1}{t2}{12}{0}{\pscirclebox[doubleline=true]{\mbox{\ \ \ \ }}} 
   \psset{arrows=->} \ncline{t2}{tD1} 
   \putnode{l1}{t2}{5}{2}{\bcar}
   \nccurve[angleA=45,angleB=135,ncurv=2,nodesep=-1]{tD1}{tD1} 
   \putnode{lD1}{tD1}{0}{7}{\acdr}
\end{pspicture} 
  }}

\newcommand{\dfaF}{%
  \scalebox{.7}{
    \psset{unit=1mm,nodesep=0mm,labelsep=0.5mm}
    \begin{pspicture}(3,-2)(24,3)
     \putnode{t0}{origin}{2}{0}{} \putnode{t1}{t0}{9}{0}{\pscirclebox{\mbox{\ \ \ \
      }}} 
       \hspace{5mm} 
        \putnode{t2}{t1}{12}{0}{\pscirclebox{\mbox{\  \ \ \  }}}
         \psset{arrows=->} \ncline{t0}{t1} \ncline{t1}{t2} \putnode{l0}{t1}{5}{2}{\clazy}
         \end{pspicture}
         }
         }

\newcommand{\dfaG}{%
  \scalebox{.7}{
    \psset{unit=1mm,nodesep=0mm,labelsep=0.5mm}
    \begin{pspicture}(3,-2)(10,3)
    \putnode{t0}{origin}{2}{0}{} \putnode{t1}{t0}{7}{0}{\pscirclebox{\mbox{\ \ \ \
      }}} \hspace{1mm} \psset{arrows=->} \ncline{t0}{t1}
      \end{pspicture}
      }
      }


\begin{wrapfigure}{l}{.25\textwidth}
\renewcommand{\arraystretch}{1}{
    \begin{uprogram}
      \UFL (\DEFINE\ (\main)
      \UNL{1} \!\!$\pi_9$:\, (\LET\  \pa\  $\leftarrow$
      (
      \scalebox{0.8}{\psframebox[framearc=.5,
	  fillcolor=lightgray,fillstyle=solid,framesep=2pt]{%
          \begin{tabular}{@{}c@{}}
            {a BIG closure}
      \end{tabular}}}
      ) \IN  
      \UNL{2} \!\!$\pi_{10}$:\, (\LET\  \pb\  $\leftarrow$ ($+$\ \pa\ \acdr)
      \IN
      \UNL{3}   \hspace*{.05cm}$\pi_{11}$:\,      (\LET\ \pc\
      $\leftarrow$  (\CONS\ \pb\ \NIL) \IN
      \UNL{4}   \hspace*{.15cm}    $\pi_{12}$:\,
      (\LET\ \pw\  $\leftarrow$  (\length\ \pc) \IN   \UNL{5} \hspace*{1.4mm}  $\pi_{13}$:\,
      (\SRETURN~$\psi_4$:\pw)))))
  \end{uprogram}}
  %%}
\caption{main program}\label{fig:main-pgm1}
\vspace{-20pt}
\end{wrapfigure}

For the main  program in Figure~\ref{fig:main-pgm1}, the liveness corresponding
to  the  variable   \var{\Lanv{\pa}{}} at $\pi_{10}$ is given by 
$\var{\Lanv{\pa}{}} \rightarrow \clazy \bcar \var{\Df{\length}{1}}$.
The  DFA   corresponding to this grammar is 
\dfaE. 
As there is no $\acar$ symbol to cancel the $\bcar$ symbol in the automaton, no 
$\epsilon$ edges are added and the $\bcar$ edge is deleted to get 
\dfaF.
We finally do the $\clazy$ simplification and notice that there is no path to a final state. 
Thus, we delete the $\clazy$ edge without marking any state as final giving the final automata for 
$\var{\Lanv{\pa}{}}$ as 
\dfaG.
The final automaton does not
accept any forward paths, reflecting  the lazy nature of our language.
Since
\length\ does not  evaluate the elements of  the argument list,
the  closure for  \pa\ is  never evaluated  and is  reclaimed whenever
liveness-based GC triggers beyond $\pi_9$.


%% \begin{wrapfigure}{r}{.25\textwidth}
%% \scalebox{0.5}{
%% \begin{tabular}{c}
%% \psset{unit=1mm,nodesep=0mm,labelsep=0.5mm}
%% \begin{pspicture}(0,-5)(50,8) %\psframe(0,-5)(40,8)
%%   \putnode{t0}{origin}{5}{0}{\var{\Lanv{\pa}{}}} \putnode{t1}{t0}{9}{0}{\pscirclebox{\mbox{\ \ \ \
%%   }}} \hspace{5mm} \putnode{t2}{t1}{12}{0}{\pscirclebox{\mbox{\  \ \ \
%%   }}}
%%   \psset{arrows=->} \ncline{t0}{t1} \ncline{t1}{t2} \putnode{l0}{t1}{5}{2}{\clazy}
%%   \putnode{tD1}{t2}{12}{0}{\pscirclebox[doubleline=true]{\mbox{\       \       \       \
%%   }}} 
%%   \psset{arrows=->} \ncline{t2}{tD1} \putnode{l1}{t2}{5}{2}{\bcar}
%%   \nccurve[angleA=45,
%%   angleB=135,                                                 ncurv=4,
%%   nodesep=-1]{tD1}{tD1} \putnode{lD1}{tD1}{0}{10}{\acdr}
%% \end{pspicture} 
%% \\
%% (a)
%% \\
%% \psset{unit=1mm,nodesep=0mm,labelsep=0.5mm}
%% \begin{pspicture}(0,-5)(50,8) %\psframe(0,-5)(40,8)
%%   \putnode{t0}{origin}{5}{0}{\var{\Lanv{\pa}{}}} \putnode{t1}{t0}{9}{0}{\pscirclebox{\mbox{\ \ \ \
%%   }}} 
%%   \hspace{5mm} 
%%   \putnode{t2}{t1}{12}{0}{\pscirclebox{\mbox{\  \ \ \  }}}
%%   \psset{arrows=->} \ncline{t0}{t1} \ncline{t1}{t2} \putnode{l0}{t1}{5}{2}{\clazy}
%% \end{pspicture}
%% \\
%% (b)
%% \\
%% \psset{unit=1mm,nodesep=0mm,labelsep=0.5mm}
%% \begin{pspicture}(0,-5)(50,8) %\psframe(0,-5)(40,8)
%%   \putnode{t0}{origin}{5}{0}{\var{\Lanv{\pa}{}}} \putnode{t1}{t0}{9}{0}{\pscirclebox{\mbox{\ \ \ \
%%   }}} \hspace{5mm} \psset{arrows=->} \ncline{t0}{t1}
%% \end{pspicture}
%% \\ 
%% (c)
%% \end{tabular}}
%%  \caption{(a) grammar  rules for \var{\Lanv{\pa}{}} converted  into an
%%   automaton and  its DFAs after simplification.}\label{fig:main1}
%% \end{wrapfigure}







 


